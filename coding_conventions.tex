\documentclass[a4paper,10pt]{article}
\usepackage[utf8]{inputenc}

\usepackage[a4paper]{geometry}
\usepackage{fullpage}

\usepackage{xspace}
\usepackage{hyperref}

\newcommand{\sificc}{SiFi-CC\xspace}
\title{Coding conventions for \sificc\ framework}

\author{\small revised for SiFi-CC by Rafał Lalik, Wojciech Kozyra, Aleksandra Wrońska\\
\small Jagiellonian University in Kraków}

\begin{document}

\maketitle

\section{References}

This document contains coding conventions suggested for the
software development for the ANKE experiment in J\"ulich, collected and revised by Volker Hejny.
To a large extend it follows some rules
which are available on the web from various collaborations,
notably CMS, ATLAS and LHCb. The major part refers to
LHCb note 2001-054 (Revised C++ Coding Conventions) by Olivier Callot.
Coding conventions for ROOT and -- to some minor extent - GEANT4 are
also taken into account.

This is the second version of this document, taking into account the
development work done during the last year.

\section{Introduction}

Software development in a collaboration implies that the result most be
usable by many people. This likely means that people, who are not the original
authors, will have to maintain them one day. Therefore, a set of rules and
conventions is necessary to insure a minimal coherence and consistency.

In the best case rules are taken mandatory. Since we will have an automatic
procedure to check them, {\bf all} people contributing to the software are
responsible to follow them.
As in many other areas of a collaboration, we must be able to rely on
the goodwill of all contributors, who are working for the success of
the overall project.

In addition, some of the rules are subject to strong personal views,
in particular the coding style, as this reflects aesthetic judgements
and are based on personal experience. Here, common sense should be used and
everyone should try to follow the majority's views. However, reasonable
exceptions are possible.

\section{Organisation}

The exact layout of the \sificc software has not been decided yet. The
modules may be redistribute to different packages and dependencies
between the packages will be resolved to some extent. The current
status reflects the first guesses at the start of the project.
Therefore, only some major points are addressed here.

The term {\bf package} describes a set of modules, which have a
particular topic in common and are put in one single library.
The term {\bf module} describes the subsets of a package. A
module is represented by a header/implementation file combination,
which in most cases contains a single class.

\begin{itemize}
\item[\bf O1]
  The \sificc code is maintained in git and can be accessed as described in
  section~\ref{sec:repos}. The git repository is {\tt \sificc}.
\item[\bf O2]
  The local configuration is done with help of the {\tt cmake} utility.
  For regular packages this means that a minimal {\tt CMakeLists.txt}
  is sufficient, which contains only a {\tt PACKAGE = name} and a
  {\tt MODULES = \dots} statement. Details are discussed within this
  guide.
\item[\bf O3]
  If one has to write an own Makefile, a {\tt Makefile.in} should be
  used in these cases, too. The top-level directory can be accessed
  via the variable {\tt @abs\_top\_srcdir}, the preprocessor, compiler
  and linker flags are represented by {\tt @CPPFLAGS@}, {\tt @CXXFLAGS@}
  and {\tt @LDFLAGS@},
  respectively.
\item[\bf O4]
  Each module has an {\bf unique name}, which should be written
  such that {\bf each word starts with an initial capital letter}. The name
  of the module is usually the name of the class it holds.
\item[\bf O5]
  Each module is represented by a header file and an implementation
  file. There should be no other ``.'' in the filename than to separate the
  extension due to portability reasons.
\item[\bf O6]
  C++ include (header) files should have the extension ``.hh'' and
  C++ implementation files should have the extension ``.cc''.
\item[\bf O7]
  For including standard files, use {\tt \#include $<$filename$>$}.
  For user files, use {\tt \#include "FileName.hh"}.\\
  There is no need to give absolute or relative paths for include files.
\end{itemize}

\section{Naming conventions}

Naming is something difficult to check by an automated tool and everybody has
his or her own feeling of what is a good name for an entity. However, one
should follow some guidelines, so that everyone understands what sort of
entity is behind the name.

\begin{itemize}
\item[\bf N1] Names should be {\bf meaningful}, but not {\bf too} long.
  Statements should stay readable!
\item[\bf N2] Names are usually made of several words, written together
  {\bf without underscores}, each first letter of a word being {\bf uppercase}.
  The case of the first letter is specified by other rules. {\bf Only
  alphanumeric characters are allowed.}\\
  Example: {\tt upperStopTime}
\item[\bf N3] Names are case sensitive. However, do {\bf not} create names
  that differ only by the case.\\
  Example: {\tt track, Track, TRACK}
\item[\bf N4] Avoid single character or meaningless names like ``jj'' except
  for local loops or local array indices.
\item[\bf N5] {\bf Class} names must be {\bf nouns} or {\bf noun phrases}.
  The first letter is capital.
\item[\bf N6] {\bf Data member} names should be private variables and start
  with ``f'' followed by an uppercase letter.\\
  - Protected variables may be used for base classes.\\
  - Never use public data members. Use accessor functions instead.
\item[\bf N7] {\bf Static variables} should start with ``g'' followed by
  an uppercase letter to distinguish them clearly.
\item[\bf N8] {\bf Avoid global variables.} Their function is better
  replaced by a class with mainly accessor methods. {\bf Avoid global
  functions and operators.} The only case where you can use them are
  symmetric binary operators and mathematical functions.\\
  If you cannot avoid a global variable, the name should start with ``g''
  followed by an uppercase letter and enclosed within {\tt SiFi} namespace.
\item[\bf N9] {\bf Constants} should start with ``k'' followed by a capital
  letter.
\item[\bf N10] {\bf Member functions} should start with a capital letter.
\item[\bf N11] {\bf Accessor functions} are named from the variable they
  control without ``f'' or ``g'' and prefixed by ``Set'' and ``Get':\\
  {\tt SetMomentum(momentum);   // set the value of fMomentum \\
       GetMomentum();           // get the value }\\
  The ``Get'' should be omitted by functions which return a non-const
  reference to indicate that this can also be used as a lvalue.
\item[\bf N12] Other functions must be {\bf verb} or {\bf verb phrases}.
\item[\bf N13] A function which creates an object and passes the responsibility
  of deleting to the caller should start with the verb {\tt Create}. The
  return value must be a pointer to the created object.\\
  If the object management remains in the class, the access should be done
  using normal accessor functions. If the user wants to indicate that a
  new object has to be created, the function should start with {\tt GetNew}.
\item[\bf N14] A function that passes the responsibility of deleting an
  object to the caller should start with {\tt Take}.
\item[\bf N15] A function that copies an object should start with {\tt Copy}.
\item[\bf N16] A function that passes the responsibility of deleting an
  object to the class should start with {\tt Adopt}. For functions which cannot
  follow that rule (e.g. constructors), the argument should contain the
  word {\tt adopt}.
\item[\bf N17] A function which deletes objects on demand should start with
  {\tt Delete}.
\end{itemize}

\section{Header files}

Header files with extension ``.hh'' contain the description of the class.
They may contain some implementation, but only in the case of the inline
code. The rest of the code is in the implementation file (``.cc''). The
general documentation of the class and the interfacing should be in
the header file.

\begin{itemize}
\item[\bf H1] A header file should contain the definition of a {\bf single
    class}. In case of nested classes both can be in the same file.
\item[\bf H2] Every header file contains {\bf a mechanism to prevent multiple
    inclusion}. The file ``IncludeFile.hh'' starts and ends by the
    following lines:%
    \begin{verbatim}
#ifndef _INCLUDEFILE_HH_
#define _INCLUDEFILE_HH_

// Here comes the code

#endif // _INCLUDEFILE_HH_
    \end{verbatim}
\item[\bf H3] One should avoid the number of header files included to avoid
  too complex dependencies. In particular, if one uses a pointer or reference
  to a class, one can just do a {\bf forward declaration} without including
  the class header file:
  \begin{verbatim}
class Line;  // forward declaration, this is enough
class Point {
public:
  Number CalcDistance(const Line &line) const;
};
  \end{verbatim}
\item[\bf H4] The class declaration starts with the ``public'' members
  first including the constructors and destructor. Then the ``protected''
  and the ``private'' sections follow.
\item[\bf H5] Header files should not have any method bodies inside the
  class definition. 
  The inline functions usually should go to
  the end of the file after the class definition. However, one could put
  {\bf a very simple accessor on the declaration line}. There is no need to 
  use the {\tt inline} keyword.\\
  Inline functions should {\bf not at all} call something external to the
  class.
\end{itemize}

\section{Implementation files}

The implementation files contain the non-trivial member functions. They
should hold the definition of a single class, they may contain auxiliary
(private) classes if needed.

\begin{itemize}
\item[\bf I1] {\bf A constructor must initialise all variables and internal
  objects.}
\item[\bf I2] {\bf A copy constructor is mandatory, together with an assignment
  operator, if a class has built-in pointer member data.}
  \begin{verbatim}
class Line {
  public:
    Line (const Line&); // copy constructor
    Line &operator= (const Line&); // assignment operator
};
  \end{verbatim}
  Note that the arguments should be {\tt const references}. If you want to
  prevent copy and/or assignment, you should provide one of the following:
  \begin{enumerate}
   \item explicite {\tt delete} keyword to copy constructor and assignment operator\\
   \begin{verbatim}
    Line(const Line&) = delete;
    Line& operator=(const Line&) = delete;
   \end{verbatim}
   \item a {\tt private} declaration of the copy and assignment constructors.
   The object can only be copied by its member and friend functions.
  \end{enumerate}
  To behave as expected, the assignment operator functions should always
  return a reference to their left operand.
\item[\bf I3] It must be guaranteed that assignment member functions work
  correctly when the left and right operands are the same object.
\item[\bf I4] {\bf Declare a ``virtual'' destructor for every class.}
  \begin{verbatim}
class Line {
  public:
    virtual ~Line();  // virtual destructor
};
  \end{verbatim}
\item[\bf I5] Virtual functions should be re-declared ``virtual'' in derived
  classes, just for clarity and to avoid mistakes when deriving a class from
  a derived class. In derived classes use the {\tt override} keyword.
\item[\bf I6] Objects should always be created in a valid, ready-to-use state
  and should not have an `open' function which must be called
  before they can be used or a `close' functions before they are deleted.
  The constructor and destructor should carry out those operations.
\item[\bf I7] Single argument non-converting constructors should use
  the {\tt explict} keyword.\\
  This is to avoid possible confusing behaviour with assignments. A
  non-converting constructor constructs objects just as converting
  constructors, but does so only where a constructor call is explicitly
  indicated by the syntax. For example with the definition of the
  constructors for Z
  \begin{verbatim}
class Z {
  public:
  explicit Z(int);
  // ...
};
  \end{verbatim}
  the following assignment would be illegal:
  \begin{verbatim}
Z a1 = 1;
  \end{verbatim}
  however
  \begin{verbatim}
Z a1(1);
  \end{verbatim}
  would work, as would:
  \begin{verbatim}
Z a1 = Z(1);
Z* p = new Z(1);
  \end{verbatim}
\item[\bf I8] {\bf Give operators conventional definitions.} For example,
  if you define the {\tt operator+}, then it should do something like addition.

\end{itemize}

\section{Member functions and arguments}
\begin{itemize}
\item[\bf M1] Functions without side effects are by far better. {\bf The use
  of the ``const'' specifier is strongly encouraged} to make clear that a
  function doesn't change the object or that the arguments are not changed.
  \begin{verbatim}
class Point {
  public:
    Number CalcDistance(const Line& line) const;
    void Translate(const Vector& vector);
};
  \end{verbatim}
  The {\tt CalcDistance} function has a {\tt const} argument, as the line is
  not modified, and is {\tt const} as the point is not affected. The
  {\tt Translate} method is also not affecting its argument, but the object
  itself is modified.
\item[\bf M2] {\bf Pass objects by constant reference.} Passing by value is
  only acceptable for basic types and pointers.
\item[\bf M3] The use of default arguments is strongly discouraged.
\item[\bf M4] {\bf Do not declare functions with unspecified (``...'')
  arguments.}
\item[\bf M5] {\bf The returned value of a function should always be tested.}
  A function that does not have a {\tt void} type should always return a
  {\bf meaningful value}.
\item[\bf M6] The exception mechanism should be used only to trap unusual
  problems.
\item[\bf M7] {\bf Always use {\tt const} modifier for non-violating class
  members.}
\end{itemize}

\section{Coding style}

This area is usually the most debated as some aesthetic considerations are
involved. There are sometimes also ``religious'' issues, because there is no
way to convince each other that this is better than that. It's only something
one can believe, but can't prove!

But, nevertheless, it is important to try to get a similar look for an easier
code maintenance. Most code writers will be replaced during the lifetime of
the detector and the source code.

\subsection{General layout}
\begin{itemize}
\item[\bf C1] The length of any line should be normally limited to 80 characters
  and never exceed 100 characters.
\item[\bf C2] Each block should be indented by two spaces. It starts by an opening
  brace on the line of the control statement and ends by an closing brace alone
  on its line. In case of longer or nested blocks a comment indicating which
  block is closed should follow the closing brace.\\
  Single line statements should only be used if the block contains only
  one statement:\\
  {\tt if (x>1.) x=1.;}
\item[\bf C3] When declaring a function with non-trivial arguments, it is
  allowed to put one argument per line. This allows inline comments.
\end{itemize}

An example of a such formatted code:
\begin{verbatim}
#include <TObject.h>

class Class : public TObject {      // opening bracket the same line
public:
  Class();
  virtual ~Class() {}               // virtual class

  void doSomething() const;         // const function
  int doSomethingElse() const { return 100; }   // single statement

private:
  Float_t val1;
  Int_t val2;
};

void Class::doSomething() const {}  // inline enclosing functions

// long arguments list
void longArgList(int a,    // this does this
                 float b,  // and this that
                 double c, // and that something else
                 char d)   // and this just is
{}

int main(int argc, char *argv[]) {

  int a = 0;
  if (0 == a)              // if statements over several lines
    a = 10;
  else if (1 == a)
    a = 20;
  else {
    a = 30;
  }

  if (0 == a) a = 100;    // if statement in single line

  switch (a) {
    case 0:
      // solve case 1 for 0
    case 1:
      break;
  }
  return 0;
}
\end{verbatim}

\subsection{Comments}
\begin{itemize}
\item[\bf C4] Use the header files to give a detailed documentation for
  purpose and usage of each class.
\item[\bf C5] Comments in the code should be short and clear. They have to
  clearify the understanding of the code, they should not repeat the code.
\item[\bf C6] The comments should be in a form that DOXYGEN can read it for
  automatic documentation.\\
  {\rm Note: This has to be discussed in detail.}
\end{itemize}

\subsection{How to write safer code}
\begin{itemize}
\item[\bf C7] {\bf Comparison between a variable and a constant should have
  the constant first.} The compiler will then spot it if you forgot one of
  the equal signs! {\tt (variable = 1)} is a valid statement,
  {\tt (1 = variable)} not!
\item[\bf C8] If you are doing an assignment within a comparison, make it
  explicit by using parenthesis.
\item[\bf C9] {\bf Comparison between floating point values should never test
  for equality.} Instead test whether the difference is smaller than an
  appropriate small number.
\item[\bf C10] To test that a pointer is valid, compare it to the value zero.
  Do not treat it as having a boolean value!
\item[\bf C11] In {\tt switch} statements {\bf each choice must have either
  a closing {\tt break}} or an explicit comment, that the fall-through is
  the desired behaviour.
\item[\bf C12] {\bf Constants must not be defined by {\tt \#define} preprocessor
  statements.} One should use {\tt enum} for integer or {\tt const} declarations.
  The best way is to put them inside a {\tt namespace} block to avoid
  naming conflicts:
  \begin{verbatim}
namespace CaloName {
  static const unsigned int kNBits = 6;
}
  \end{verbatim}
  and use them like:
  \begin{verbatim}
if (CaloName::kNBits == variable) { ... }
  \end{verbatim}
\item[\bf C13] {\bf Use inlines instead of function macros.} {\bf Don't use
  function macros}, they are problematic. Instead, declare the functions
  inline to obviate the need for function macros.
  Like const, inline functions follow the C++ scope rules and allow argument
  type-checking. Both member functions and nonmember functions can be
  declared inline. Consider this classic example:
  \begin{verbatim}
#define SQUARE(x) ((x)*(x))

// and...

SQUARE(y++);        // y incremented twice
  \end{verbatim}
  When written as an inline, it is actually more efficient than the macro
  version. What's more, it's correct.
  \begin{verbatim}
inline int Square(int x) {
  return x*x;
};

Square(y++);        // y incremented once
  \end{verbatim}
\item[\bf C14] Re-use existing classes. {\bf STL should be exploited.} Old
  C habits should be changed ...
  \begin{itemize}
    \item Use root types for all class members and arguments of methods.
    \item Use {\tt std::string} or {\tt TString} instead of {\tt char*}.
    \item Don't use built-in arrays. Use one of the STL containers like
          {\tt std::vector}.
    \item Use standard classes line {\tt TVector3} rather than {\tt double v[3]}.
    \item Do not use {\tt union} types.
    \item Use {\tt Bool\_t} for logical values.
  \end{itemize}
\item[\bf C15] Avoid overloading operators unless there is a clear improvement
  in the clarity of the code.
\item[\bf C16] {\bf Avoid complicated implicit precedence rules}, use parenthesis
  to clarify your wishes.
\item[\bf C17] {\bf Use cast operators} for data-type conversion. You should use
  {\tt static\_cast} or {\tt dynamic\_cast}, but not {\tt reinterpret\_cast} or
  {\tt const\_cast}. The best is to provide the appropriate cast operator inside
  the class.
\item[\bf C18] {\bf All C++ entities should be defined only in the smallest
    scope needed.} Although there is a problem of inconsistency between some
  compilers, the use of\\
  {\tt for (int j=0; 10<j; j++) \{ ... \}}\\
  is recommended.
\item[\bf C19] A function must not use the delete operator to any pointer
  passed to it as argument.
\item[\bf C20] Use {\tt new} and {\tt delete} in place of {\tt malloc()} and
  {\tt free()}.
\item[\bf C21] Any pointer to automatic objects should have the same or a
  smaller scope than the object it points to.
\item[\bf C22] Just to mention it: Don't use {\tt goto} ;-).
\end{itemize}

\section{Documenting code}

\begin{itemize}
  \item[\bf D1] Use Qt-like style of Doxygen to document class, members and
    standalone functions. It starts with multiline comment {\tt /*} followed by
    single an exclamation mark {\tt !} and ends with {\tt */} in the at least
    next line with asteriks {\tt *} alligned each other. Intermediate lines must
    contain aligned asteriks. An example:
    \begin{verbatim}
/*!
 * ... text ...
 */
    \end{verbatim}
  \item[\bf D2] Each function must have documented purpose, params and return
    value using {\tt \textbackslash{}param} and {\tt \textbackslash{}return}
    keywords of Doxygen. An example:
    \begin{verbatim}
/* Function which does nothing but serves as an example.
 * \param bar first parameter to do nothing
 * \return value to do nothing but must be checked (see M5).
 */
int foo(float bar) { return 13; }
    \end{verbatim}
    Class members must also be documented.
  \item[\bf D3] If desired, use {\tt \textbackslash{}sa} to point to other
    methods in the documentation.
  \item[\bf D4] For details about documenting the code refer to the Doxygen
    manual: \url{http://www.doxygen.org/manual/docblocks.html}.
\end{itemize}


\section{\sificc specific rules}

This points mentioned here are related to the \sificc software.

\begin{itemize}
\item[\bf A1] Do not use C++ standard streams. Include {\tt CLog.hh} and use
  {\tt\bf gScreen} and {\bf\tt gLog} instead.
\item[\bf A2] If an index refers to a detector or any other countable, physical
  object (like wires), the index should start at 1.
\item[\bf A3] For data storage, ROOT containers should be prefered. To allow
 access to such a container in your data object, offer the number
 of entries ({\tt GetXXXEntries()}) and the container itself
({\tt GetXXX()}). Avoid to implement index operators by yourself.
\end{itemize}

\section{Code repositories}\label{sec:repos}

TODO

\end{document}
